%! Author vonesch, wächter,sebrek
%! Date = 11.12.21

% Preamble
\documentclass[11pt]{article}
\usepackage[letterpaper,top=1.5cm,bottom=2cm,left=2cm,right=2cm,marginparwidth=1.75cm]{geometry}

% Packages
\usepackage{amsmath}
\title{Bloom Filter}
\author{Tobias Vonesch, Michelle Wächter, Aleks Sebrek}

% Document
\begin{document}
    \maketitle

    \section{Idee des Bloom-Filters}

    Ein Bloom-Filter ist eine, auf Hashing basierende probabilistische Datenstruktur.
    Er wird in der Regel verwendet, um Elemente zu einer Menge hinzuzuf{\"u}gen und zu pr{\"u}fen, ob ein Element in einer Menge enthalten ist, da er extrem Platzsparend ist.
    Nicht die Elemente selbst werden zu einer Menge hinzugef{\"u}gt, sondern der Hash der Elemente.
    \smallskip
    Bei der {\"U}berpr{\"u}fung, ob ein Element im Bloom-Filter enthalten ist, ist es m{\"o}glich false positives zu erhalten.
    Entweder stellt man fest, dass ein Element definitiv nicht in der Menge enthalten ist, oder dass es m{\"o}glich ist, dass das Element in der Menge enthalten ist.
    Wenn der Bloom-Filter allerdings sagt, dass das Element nicht in der Datenstruktur enthalten ist, ist es ganz bestimmt nicht darin enthalten.
    \smallskip
    Ein Bloom-Filter {\"a}hnelt einer Hash-Tabelle, da er eine Hash-Funktion verwendet, um einen Schl{\"u}ssel einem Bucket zuzuordnen.
    Er speichert diesen Schl{\"u}ssel jedoch nicht in diesem Bereich, sondern markiert ihn lediglich als gef{\"u}llt.
    So k{\"o}nnen viele Schl{\"u}ssel demselben gef{\"u}llten Bucket zugeordnet werden, was zu falsch positiven Ergebnissen f{\"u}hrt.

    \subsection{Anforderung Hash-Funktion}
    Die in einem Bloom-Filter verwendeten Hash-Funktionen sollten unabh{\"a}ngig und gleichm{\"a}ssig verteilt sein.
    Sie sollten ausserdem so schnell wie mo{\"o}lich sein (kryptografische Hash-Funktionen wie sha1, sind daher keine gute Wahl).

    \subsection{Grösse Bloom-Filter}
    Eine Eigenschaft von Bloom-Filtern ist, dass man die false positive-Rate des Filters ver{\"a}ndern kann.
    \newline
    Ein gr{\"o}sserer Filter sprich mehr Hash-Funktionen ist Speicher- und Rechenintensiver, ein kleiner Filter und somit weniger Hash-Funktionen ist daf{\"u}r fehleranf{\"a}lliger.
    Es gilt je nach Anwendungszweck eine gute Balance zu finden.
    \newline
    Die false positive-Rate betr{\"a}gt ungef{\"a}hr:
    \begin{align*}
    P = (1-e^{-kn/m})^k
    \end{align*}
    Man kann also die Anzahl n der Elemente eingeben, die eingef{\"u}t werden sollen, und verschiedene Werte f{\"u}r k und m ausprobieren, um den Filter zu konfigurieren.

    \subsection{Anzahl Hash-Funktionen}
    Je mehr Hash-Funktionen, desto langsamer ist der Bloom-Filter und desto schneller f{\"u}llt er sich.
    Wenn man jedoch zu wenige oder zu viele hat, kann es zu vielen false positives Ergebnissen kommen.
    \smallskip
    Da bei der Erstellung des Filters die Anzahl Hash-Funktionen k gew{\"a}hlt werden muss, muss vorher abgesch{\"a}tzt werden, in welchem Bereich Gr{\"o}sse Datensatz n erwartet wird.
    F{\"u}r dieses Optimierungsproblem gibt es f{\"u}r m eine Funktion, mit der man den optimalen Wert k ermitteln kann:
    \begin{align*}
    k = (\frac{m}{n})\left ln2
    \end{align*}
    \newline
    k = Anzahl Filter Funktionen
    \newline
    m = die Gr{\"o}sse des Filters
    \newline
    n = die Gr{\"o}sse des Datensatzes
    \newline
    \subsection{Vor- und Nachteile}
    \subsubsection{Vorteile}
    \begin{itemize}
        \item Hash-Funktionen k{\"o}nnen Simultan berechnet werden
        \item Anfragen k{\"o}nnen gleichzeitig gestellt werden, bis ein neues Element hinzugef{\"u}gt wird
        \item geringer Overhead durch effiziente Datenstruktur (Array)
    \end{itemize}

    \subsubsection{Nachteile}
    \begin{itemize}
        \item Zus{\"a}tzliche Datenstruktur notwendig, welche sehr gross werden kann
        \item Hash-Wertberechnung bei jedem Schreib- und Lesevorgang
    \end{itemize}

    \section{Beispiel aus der Praxis}
    Der Webbrowser Google Chrome verf{\"u}gt {\"u}ber einen Bloomfilter mit den Signaturen gef{\"a}hrlicher Webseiten und {\"u}berpr{\"u}ft bei Eingabe einer URL, ob diese in diesem Filter enthalten ist.
    Falls dies der Fall ist, wird dem User eine Warnung pr{\"a}sentiert, welche ihn {\"u}ber die potentielle Gef{\"a}hrlichkeit der zu besuchenden Website aufkl{\"a}ren soll.

    \section{Fehlerwahrscheinlichkeit}
    In unserem Programm ist es m{\"o}glich eine Fehlerwahrscheinlichkeit p vorzugeben.
    Mithilfe des vorgegebenen p und der Gr{\"o}sse des Datensatzes werden Filter und Anzahl Hashes berechnet.
    Die Gr{\"o}sse des Filters wird mittels der Folgenden Formel bestimmt:

     \begin{align*}
        m = \frac{n * log(p)}{log(\frac{1}{2^{log(2)}})}
    \end{align*}
    \newline
    m = die Gr{\"o}sse des Filters
    \newline
    n = die Gr{\"o}sse des Datensatzes
    \newline
    p = gew{\"u}nschte Probabilit{\"a}t
    \medskip
    \newline
    Die Anzahl der Hashes wird mittels folgender Formel bestimmt:

    \begin{align*}
        k = -(log(p) * log(2))
    \end{align*}
    \smallskip
    k = Anzahl der Hash-Funktionen (aber mindestens 1)
    \\
    p = gew{\"u}nschte Probabilit{\"a}t
    \medskip

    \subsection{Test}
    Um auszuwerten wie gut unser Filter funktioniert und wie hoch die effektive FalsePositive Rate ist,
    haben wir einen zweiten, komplett verschiedenen Datensatz mit W{\"o}rtern der Niederl{\"a}ndischen Sprache zusammengesucht.
    So l{\"a}sst sich einfach sagen, dass jedes Positive ein FalsePositive sein muss. Der zweite Datensatz ist {\"u}berdies um einiges
    gr{\"o}sser als der vorgegebene. Um die Effektivit{\"a}t zu messen, teilen wir also einfach die Anzahl FalsePositives durch die Anzahl
    Testw{\"o}rter.

    \begin{align*}
    P = (\frac{False Positives}{Anzahl Woerter})
    \end{align*}

    \newpage
    \subsection{Resultate}
    Nach ausf{\"u}rlichen Tests haben wir festgestellt, dass der Algorithmus verh{\"a}ltnissm{\"a}ssig  gut f{\"u}r p-Werte gr{\"o}sser gleich 0.1 funktioniert.
    \newline
    Bei P = 0.9, 0.7, 0.5 betr{\"a}gt die Abweichung weniger als 10\%.
    \newline
    Bei P kleiner als 0.1 wird die Abweichung immer gr{\"o}sser mit +80\% bei P = 0.03, +240\% bei P = 0.01 und +488\% bei P = 0.005.
    \newline
    Die Funktion von P flacht also deutlich ab.
    \medskip
    \newline
    Ersichtlich in folgender Tabelle:
    \newline
    \medskip
    \begin{table}[htp]
        \label{tab:summary}
        \begin{center}
            \begin{tabular}{ccccc}
                \hline
                &                 &             &Nummer     \\
                Selected P  & Actual P        & Filter Size &Hash       \\
                &                 &             &Functions	\\
                \hline

                0.9         & 0.9900          &  12743       & 1  	\\
                0.7         & 0.7469          &  43139       & 1  	\\
                0.5         & 0.5116          &  83835       & 1  	\\
                0.3         & 0.3499          &  145618      & 1  	\\
                0.1         & 0.1224          &  278493      & 3  	\\
                0.09        & 0.1122          &  291237      & 3  	\\
                0.05        & 0.0745          &  362328      & 4    \\
                0.03        & 0.0543          &  424112      & 5  	\\
                0.01        & 0.0345          &  556987      & 6  	\\
                0.005       & 0.0294          &  640822      & 7  	\\
                \hline
            \end{tabular}
            *Nummer of Hash-Functions
        \end{center}
        \caption{Die Filtergenauigkeit bei gleichem Datensatz aber anderem P}
    \end{table}




\end{document}
