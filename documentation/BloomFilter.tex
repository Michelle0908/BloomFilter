%! Author vonesch, wächter,sebrek
%! Date = 11.12.21

% Preamble
\documentclass[11pt]{article}
\usepackage[letterpaper,top=2cm,bottom=2cm,left=2cm,right=2cm,marginparwidth=1.75cm]{geometry}


% Packages
\usepackage{amsmath}

\title{Bloom Filter}
\author{Vonesch, Wächter, Sebrek}

% Document
\begin{document}
    \maketitle

    \section{Idee des Bloom-Filters}

    Ein Bloom-Filter ist eine, auf Hashing basierende probabilistische Datenstruktur.
    Er wird in der Regel verwendet, um Elemente zu einer Menge hinzuzuf{\"u}gen und zu pr{\"u}fen, ob ein Element in einer Menge enthalten ist, da er extrem Platzsparend ist.
    Nicht die Elemente selbst werden zu einer Menge hinzugef{\"u}gt, sondern der Hash der Elemente. \\

    Bei der {\"U}berpr{\"u}fung, ob ein Element im Bloom-Filter enthalten ist, ist es m{\"o}glich false positives zu erhalten.
    Entweder stellt man fest, dass ein Element definitiv nicht in der Menge enthalten ist, oder dass es m{\"o}glich ist, dass das Element in der Menge enthalten ist.
    Wenn der Bloom-Filter allerdings sagt, dass das Element nicht in der Datenstruktur enthalten ist, ist es ganz bestimmt nicht darin enthalten.\\

    Ein Bloom-Filter {\"a}hnelt einer Hash-Tabelle, da er eine Hash-Funktion verwendet, um einen Schl{\"u}ssel einem Bucket zuzuordnen.
    Er speichert diesen Schl{\"u}ssel jedoch nicht in diesem Bereich, sondern markiert ihn lediglich als gef{\"u}llt.
    So k{\"o}nnen viele Schl{\"u}ssel demselben gef{\"u}llten Bucket zugeordnet werden, was zu falsch positiven Ergebnissen f{\"u}hrt.

    \subsection{Anforderung Hash-Funktion}
    Die in einem Bloom-Filter verwendeten Hash-Funktionen sollten unabh{\"a}ngig und gleichm{\"a}ssig verteilt sein.
    Sie sollten ausserdem so schnell wie mo{\"o}lich sein (kryptografische Hash-Funktionen wie sha1, sind daher keine gute Wahl).

    \subsection{Vorteile}

    \subsection{Nachteile}

    \section{Beispiel aus der Praxis}

    \section{Fehlerwahrscheinlichkeit}

    \subsection{Test}

    \subsection{Resultate}



\end{document}
